\section{Exercice 1 }

\subsection{Méthode des moments}

\subsubsection{Echantillon}
cf. code \texttt{Matlab}

\subsubsection{Estimateur}

On cherche à déterminer un estimateur 

Vu la forme de la loi de densité 
$$f(x) = \frac{k}{c} (\frac{x}{c})^{k-1} \exp[(\frac{x}{c})^{k}]$$

on sait qu'il s'agit d'une distribution de Weibull et que donc : 

$$ \mu'_1 = E(x) = c \Gamma(1+\frac{1}{k})$$

De plus, par définition, 

$$V(x) = E(x^2) - (E(x))^2$$

Or pour une Weibull, on a que 

$$V(x) = c^2 [\Gamma(1+\frac{2}{k}) - (\Gamma(1+\frac{1}{k}))^2]$$.

On déduit donc aisément que 

$$ \mu'_2 = E(x^2) = c^2 \Gamma(1+\frac{2}{k})  $$

De là on tire que : 

$$\frac{\mu'_1}{\Gamma(1+\frac{1}{k})} = c $$
$$\frac{\mu'_2}{\mu_{1}^{'2}} = \frac{\Gamma(1+\frac{2}{k})}{(\Gamma(1+\frac{1}{k}))^2}$$

Ce qui donne, par propriété de la fonction $\Gamma$ : 
$$\frac{\mu'_2}{\mu_{1}^{'2}} = \frac{2 k \Gamma(\frac{2}{k})}{(\Gamma(\frac{1}{k}))^2}$$ 

Ensuite on égalise nos $\mu$ et $\hat{\mu}$

$$\frac{\hat{\mu}'_2}{\hat{\mu}_{1}^{'2}} = \frac{2 k \Gamma(\frac{2}{k})}{(\Gamma(\frac{1}{k}))^2}$$ 
$$\frac{ n \sum_i^n X_i^2}{(\sum_i^{n} X_i)^2} = \frac{2 k \Gamma(\frac{2}{k})}{(\Gamma(\frac{1}{k}))^2}$$ 

On procède ensuite par itération pour trouver la valeur idéale de $k$ et à partir de laquelle on peut déduire $c$

On obtient des valeurs de k et c. 